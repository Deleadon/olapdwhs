% Präambel
\documentclass[12pt,a4paper,oneside, 
liststotoc, 					% Tabellen- und Abbildungsverzeichnis ins Inhaltsverzeichnis
bibtotoc,						% Literaturverzeichnis ins Inhaltsverzeichnis aufnehmen
titlepage, 						% Titlepage-Umgebung statt \maketitle
headsepline, 					% horizontale Linie unter Kolumnentitel
%abstracton,					% Überschrift beim Abstract einschalten, Abstract muss dazu in {abstract}-Umgebung stehen
%DIV11,							% auskommentieren, um den Seitenspiegel zu vergrößern
BCOR6mm,						% Bindekorrektur, die den Seitenspiegel um 6mm nach rechts verschiebt,
openany,							% Unterdrückung von leeren Seiten nach Chapter-Ende
]{scrreprt}			
%\usepackage{utf8} 				% Dokument in utf8-Codierung schreiben und speichern
\usepackage[utf8]{inputenc} 	% ermöglicht die direkte Eingabe von Umlauten
\usepackage[paper=a4paper,left=30mm,right=30mm,top=25mm,bottom=25mm]{geometry} % seitenabstand
\usepackage[ngerman]{babel} 	% deutsche Trennungsregeln und Übersetzung der festcodierten Überschriften
\usepackage[T1]{fontenc} 		% Ausgabe aller zeichen in einer T1-Codierung (wichtig für die Ausgabe von Umlauten!)
\usepackage{graphicx}  			% Einbinden von Grafiken erlauben
%\usepackage{amsmath}
%\usepackage{amsfonts}
%\usepackage{amssymb}
\usepackage{mathpazo} 			% Einstellung der verwendeten Schriftarten
\usepackage{textcomp} 			% zum Einsatz von Eurozeichen u. a. Symbolen
\usepackage{listings}			% Datstellung von Quellcode mit den Umgebungen {lstlisting}, \lstinline und \lstinputlisting
\usepackage{xcolor} 			% einfache Verwendung von Farben in nahezu allen Farbmodellen
\usepackage[intoc]{nomencl} 	% zur Erstellung des Abkürzungsberzeichnisses
\usepackage{fancyhdr}			% Zusatzpaket zur Gestaltung von Fuß und Kopfzeilen
% persönliche Packages:
\usepackage{subfigure} 			% 2 Bilder nebeneinander
\usepackage{placeins}
\usepackage[onehalfspacing]{setspace} % Zeilenabstand
% Litaraturverzeichnis:
\usepackage[backend=bibtex,style=authoryear-icomp]{biblatex}
\usepackage[babel,german=guillemets]{csquotes}
\usepackage{acronym} % Abkürzungsverzeichnis
\usepackage{glossaries}
\bibliography{Inhalt/literatur.bib}
\nocite{*}
% Java Code:
\usepackage{color}

\definecolor{dkgreen}{rgb}{0,0.6,0}
\definecolor{gray}{rgb}{0.5,0.5,0.5}
\definecolor{mauve}{rgb}{0.58,0,0.82}

\lstset{frame=tb,
  language=Java,
  aboveskip=3mm,
  belowskip=3mm,
  showstringspaces=false,
  columns=flexible,
  basicstyle={\small\ttfamily},
  numbers=none,
  numberstyle=\tiny\color{gray},
  keywordstyle=\color{blue},
  commentstyle=\color{dkgreen},
  stringstyle=\color{mauve},
  breaklines=true,
  breakatwhitespace=true,
  tabsize=3
}
% -----------------------------------------------------------------------------------------------------------------
% Zum Aktualisieren des Abkürzungsverzeichnisses bitte auf der Kommandozeile folgenden Befehl aufrufen :
%  makeindex Bachelorarbeit.nlo -s nomencl.ist -o Bachelorarbeit.nls 
%  

% Schneller Compiler (bibtex, akürzungen, latex, pdf anzeigen im internen fenster (texmaker))
% 	makeindex Bachelorarbeit.nlo -s nomencl.ist -o Bachelorarbeit.nls | bibtex %|pdflatex -synctex=1 -interaction=nonstopmode %.tex|"C:/Program Files (x86)/Adobe/Reader 11.0/Reader/AcroRd32.exe" %.pdf
% 
% einfach in Bachelorarbeit.tex ausführen -----------------------------------------------------------------------------------------------------------------

%usepackage float mit H für Bilder

% Hier die persönlichen Daten eingeben:

\newcommand{\titel}{Frontend Webentwicklung}
\newcommand{\untertitel}{eines Internet of Things Showcases am Beispiel von Connected Cars}
\newcommand{\arbeit}{Praxisbericht}
\newcommand{\prufungvortext}{T2000}
\newcommand{\prufung}{Bericht über PE 3/4} 
\newcommand{\studiengang}{Angewandte Informatik}
\newcommand{\autor}{Robin Schlenker}
\newcommand{\matrikelnr}{2006895}
\newcommand{\kurs}{TINF14A}
\newcommand{\firma}{IBM Deutschland GmbH}
\newcommand{\abgabe}{\today}
\newcommand{\betreuerdhbw}{unbekannt}

\newcommand{\jahr}{2015}			% für Angabe im Copyright-Vermerk der Titelseite

% Abkürzungen
\newcommand{\ua}{\mbox{u.\,a.\ }}
\newcommand{\zb}{\mbox{z.\,B.\ }}
\newcommand{\bs}{$\backslash$}

\renewcommand{\nomname}{Abkürzungsverzeichnis}

% -------------------------------------------------------------------------------------------
% Definition der Kopf- und Fußzeilen
\lhead{}								% Kopf links
\chead{}								% Kopf mitte
\rhead{\sffamily{Frontendentwicklung IoT ConnectedCars}}				% Kopf rechts
\lfoot{\prufungvortext}					% Fuß links
\cfoot{\sffamily{\thepage}}				% Fuß mitte
\rfoot{\sffamily{\autor}}				% Fuß rechts
\renewcommand{\headrulewidth}{0.4pt}	% Liniendicke Kopf
\renewcommand{\footrulewidth}{0.4pt}	% Liniendicke Fuß

\newglossaryentry{ETL}
{
  name=Extraction Transform and Load,
  description={Process which desribes how the data from the databases is accessed, transformed and stored into the datawarehouse}
}

\makeglossaries
% -------------------------------------------------------------------------------------------
%                     Beginn des Dokumenteninhalts
% -------------------------------------------------------------------------------------------
\begin{document}
\setcounter{secnumdepth}{4}					% Nummerierungstiefe fürs Inhaltsverzeichnis
\setcounter{tocdepth}{2}
\sffamily									% für die Titelei serifenlose Schrift verwenden

% ------------------------------ Titelei -----------------------------------------------------

\thispagestyle{plain}
\begin{titlepage}
\enlargethispage{4.0cm}
\sffamily 								% Serifenlose Grundschrift für die Titelseite einstellen

\singlespacing
\begin{figure}
	\hspace{-2.0cm}
     \raisebox{0.4cm}[1cm]{\subfigure{\includegraphics[scale=1.5]{Bilder/logo_ibm.jpg}\\[5ex]}}
    \hspace{7cm}
    \subfigure{\includegraphics[scale=2.0]{Bilder/logo_dhbw.jpg}\\[5ex]}
\end{figure} 


\begin{center}

\huge{\textsc{\textbf{\titel}}}\\[1.5ex]
\Large{\textbf{\untertitel}}\\[5ex]
\LARGE{\textbf{\arbeit}}\\[2ex]
\normalsize{\prufungvortext \\[1ex] \prufung}\\[3ex]
\Large{Studiengang \studiengang}\\[1ex]
\normalsize{Duale Hochschule Baden-Württemberg Stuttgart}\\[5ex]
von\\[1ex] \autor \\[2ex] \small{\textit{Bitte beachten Sie den Sperrvermerk!}} \\[14ex]

\end{center}

\begin{flushleft}

\begin{tabular}{ll}
Abgabedatum:					& \quad \abgabe \\
%Bearbeitungszeitraum:			& \quad 12 Wochen   \\ 
Matrikelnummer, Kurs: 			& \quad \matrikelnr , \kurs \\ 
Ausbildungsfirma:	 			& \quad \firma \\ 
Gutachter der Dualen Hochschule: & \quad \betreuerdhbw \\ [5ex]

\end{tabular} 



\small
Copyrightvermerk:\\

Dieses Werk einschließlich seiner Teile ist \textbf{urheberrechtlich geschützt}. Jede Verwertung außerhalb der engen Grenzen des Urheberrechtgesetzes ist ohne Zustimmung des Autors unzulässig und strafbar. Das gilt insbesondere für Vervielfältigungen, Übersetzungen, Mikroverfilmungen sowie die Einspeicherung und Verarbeitung in elektronischen Systemen.
\end{flushleft}
\begin{flushright}
\copyright{} \jahr
\end{flushright}
\end{titlepage} 				% erzeugt die Titelseite
\pagenumbering{Roman}						% große, römische Seitenzahlen für Titelei	
\addchap*{Eidesstattliche Erklärung}
Ich versichere hiermit, dass ich meine Arbeit mit dem Thema
\begin{quote}
\textit{\titel} -\textit{ \untertitel }
\end{quote}
selbständig verfasst und keine anderen als die angegebenen Quellen und Hilfsmittel benutzt habe. Die Arbeit wurde bisher keiner anderen Prüfungsbehörde vorgelegt und auch nicht veröffentlicht.  \\[10ex]


% Mir ist bekannt, dass ich meine Diplomarbeit zusammen mit dieser Erklärung fristgemäß nach Vergabe des Themas in dreifacher Ausfertigung und gebunden im Sekretariat meines Studiengangs an der DHBW Karlsruhe abzugeben habe. Als Abgabetermin giltbei postalischer Übersendung der Eingangsstempel der DHBW, also nicht der Poststempel oder der Zeitpunkt eines Einwurfs in einen Briefkasten der DHBW.


Stuttgart, den \today \\[4ex]

%\includegraphics[scale=1]{unterschrift.png}\\[-5ex]
\rule[0.2cm]{8cm}{0.5pt} \\
\textsc{\autor} ~ \textit{ (\matrikelnr)} \\[10ex]

% Sperrvermerk bei Bedarf dekommentieren
\hrule 
\vspace*{1.0cm}
\noindent \textbf{\Large{Sperrvermerk}}\\
\normalsize
\textbf{Die Ergebnisse der Arbeit stehen ausschließlich dem auf dem Deckblatt aufgeführten Ausbildungsbetrieb, der \firma , zur Verfügung.} 				% Einbinden der eidestattlichen Erklärung

\singlespacing
\tableofcontents							% Erzeugen des Inhalsverzeichnisses
\onehalfspacing
\printnomenclature[2.0cm]					% Erzeugen des Abkürzungsverzeichnisses
\listoffigures 								% Erzeugen des Abbildungsverzeichnisses 

\listoftables 								% Erzeugen des Tabellenverzeichnisses
\pagebreak

\pagenumbering{arabic}						% arabische Seitenzahlen für den Hauptteil
\pagestyle{fancy}					
\rmfamily

\chapter{Introduction}

\chapter{ Definition }\label{definition}
\section{Data Warehouse}\label{dw}
\section{OLAP}\label{olap}

\chapter{Use cases}\label{usecases}
\section{Business Analysis}\label{ba}
\chapter{Software products}\label{software}
  
% ---------------------------- Literaturverzeichnis ----------------------------------------------
\printbibliography

% ------------------------------- Anhang ---------------------------------------------------------

\begin{appendix}
\clearpage
\pagenumbering{Roman}						% römische Seitenzahlen für Anhang
\end{appendix}


\end{document}
