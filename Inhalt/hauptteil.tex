\chapter{Grundlagen}
\label{cha:Grundlagen}
Das Messegeschäft, insbesondere die CeBIT, ist für die IBM eines der wesentlichen Marketing- und Geschäftsrepräsentierungs Events. Um diese effektiv zu nutzen ist eine klare firmeninterne Informationsweitergabe und -verwendung essentiell wichtig. Um ein Grundverständnis aufzubauen wie Daten und Kontakte weiterführend verwendet werden, erklären die folgenden Unterkapitel verschiedene Begriffe. 
\section{Lead}
Geschäftskontakte sind der Grundstein eines jeden erfolgreichen Unternehmens. Ein Lead bezeichnet einen Kontakt zwischen einem möglichen Neukunden und einem Unternehmen. Der Kunde hat dabei freiwillig seine Kontaktdaten \zb einem Mitarbeiter an einem Messestand gegeben oder in ein Onlineformular eingetragen. Innerhalb der IBM gelten jedoch nicht nur neue Kontakte als Lead, sondern auch Gespräche mit Bestandskunden oder Interesse von anderen Mitarbeitern einer bereits in Beziehung stehenden Firma. Auch Kontakte von Business Partnern (kooperierende Firmen, mit denen IBM zusammenarbeitet) gelten als Lead.\\
Ein \glqq Qualified Lead\grqq ~beschreibt einen Kontakt mit einem sehr interessierten Kunden, bei dem die Chancen auf ein Geschäftsverhältnis hoch sind.
\section{Opportunity (Opty)}
% parenthese
Eine Opty ist -- wie der Name bereits ausdrückt -- eine bestimmte Möglichkeit ein Geschäftsverhältnis aufzubauen. Hierbei gilt ein Lead als Basis, woraus eine oder mehrere Optys entstehen können. Ein Geschäftskontakt wird  auf ein bestimmtes Produkt oder Service spezifiziert, sodass firmeninterne Ansprechpartner den Lead erhalten. Somit können Mitarbeiter aus einer oder mehreren Abteilungen zusammen die Bedürfnisse und Wünsche eines möglichen Kunden angehen und im besten Fall über ein Produkt oder einen Service befriedigen.
\section{Report}
Die Analyse von gesammelten Firmendaten ist eines der wichtigsten Managementinstrumente. Zu diesen Informationen gehören unter anderem auch die Kennzahlen von Kundenkontakten und deren Progression. Ein solcher Bericht, auch Report genannt, sollte möglichst zielgerichtet spezifische, für das Unternehmen wichtige Informationen liefern.\\
Als Ausgangsbasis werden dabei Daten aus dem aktiven Operativen Systemen des Unternehmens genommen \footcite{rob}. Die Art und der Informationsgehalt eines Reports hängt dabei stark von Unternehmen und Branche ab. Entscheidend ist, dass das höhere Management aufgrund dieser Daten grundlegende Entscheidungen trifft und somit eine hohe Qualität der Berichte erforderlich ist.




\textbf{}
%\begin{figure}[htb]
%\centering
%\includegraphics[width=0.8\textwidth]{FHWTLogo.jpg}
%\caption{Das Logo der FHWT}
%\label{fig:FHWTLogo}
%\end{figure}

